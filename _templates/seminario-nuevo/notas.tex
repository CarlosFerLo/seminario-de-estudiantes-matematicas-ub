\documentclass[12pt,a4paper]{article}
\usepackage[utf8]{inputenc}
\usepackage[spanish]{babel}
\usepackage{amsmath,amsthm,amssymb}
\usepackage{tikz-cd}
\usepackage{hyperref}

\title{Título del Seminario}
\author{Nombre del Ponente \\ Seminario de Estudiantes de Matemáticas UB}
\date{Semestre Año}

\theoremstyle{definition}
\newtheorem{definition}{Definición}[section]
\newtheorem{example}{Ejemplo}[section]
\theoremstyle{plain}
\newtheorem{theorem}{Teorema}[section]
\newtheorem{lemma}[theorem]{Lema}
\newtheorem{proposition}[theorem]{Proposición}
\newtheorem{corollary}[theorem]{Corolario}

\begin{document}

\maketitle

\tableofcontents

\section{Introducción}

% Tu contenido aquí

\section{Definiciones básicas}

\begin{definition}
% Definición
\end{definition}

\section{Resultados principales}

\begin{theorem}
% Teorema
\end{theorem}

\begin{proof}
% Demostración
\end{proof}

\section{Ejemplos}

\begin{example}
% Ejemplo
\end{example}

\end{document}
