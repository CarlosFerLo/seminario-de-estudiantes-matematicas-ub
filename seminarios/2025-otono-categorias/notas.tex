\documentclass[12pt,a4paper]{article}
\usepackage[utf8]{inputenc}
\usepackage[spanish]{babel}
\usepackage{amsmath,amsthm,amssymb}
\usepackage{tikz-cd}
\usepackage{hyperref}

\title{Introducción a la Teoría de Categorías}
\author{Estudiante 1 \\ Seminario de Estudiantes de Matemáticas UB}
\date{Otoño 2025}

\theoremstyle{definition}
\newtheorem{definition}{Definición}[section]
\newtheorem{example}{Ejemplo}[section]
\theoremstyle{plain}
\newtheorem{theorem}{Teorema}[section]
\newtheorem{lemma}[theorem]{Lema}

\begin{document}

\maketitle

\tableofcontents

\section{Categorías}

\begin{definition}
Una \textbf{categoría} $\mathcal{C}$ consiste en:
\begin{enumerate}
    \item Una colección de \textbf{objetos} $\text{Ob}(\mathcal{C})$
    \item Para cada par de objetos $A, B$, un conjunto de \textbf{morfismos} $\text{Hom}(A, B)$
    \item Una \textbf{composición} $\circ : \text{Hom}(B, C) \times \text{Hom}(A, B) \to \text{Hom}(A, C)$
\end{enumerate}
\end{definition}

\begin{example}
La categoría $\mathbf{Set}$ tiene como objetos todos los conjuntos y como morfismos las funciones.
\end{example}

\section{Funtores}

\begin{definition}
Un \textbf{funtor} $F: \mathcal{C} \to \mathcal{D}$ asigna:
\begin{itemize}
    \item A cada objeto $A \in \mathcal{C}$ un objeto $F(A) \in \mathcal{D}$
    \item A cada morfismo $f: A \to B$ un morfismo $F(f): F(A) \to F(B)$
\end{itemize}
preservando identidades y composición.
\end{definition}

\section{Transformaciones Naturales}

\begin{definition}
Una \textbf{transformación natural} $\eta: F \Rightarrow G$ entre funtores $F, G: \mathcal{C} \to \mathcal{D}$ es una familia de morfismos $\eta_A: F(A) \to G(A)$ tal que el siguiente diagrama conmuta:
\[
\begin{tikzcd}
F(A) \arrow[r, "\eta_A"] \arrow[d, "F(f)"'] & G(A) \arrow[d, "G(f)"] \\
F(B) \arrow[r, "\eta_B"'] & G(B)
\end{tikzcd}
\]
\end{definition}

\section{El Lema de Yoneda}

\begin{theorem}[Lema de Yoneda]
Para todo funtor $F: \mathcal{C} \to \mathbf{Set}$ y objeto $A \in \mathcal{C}$:
\[
\text{Nat}(\text{Hom}(A, -), F) \cong F(A)
\]
\end{theorem}

\end{document}
